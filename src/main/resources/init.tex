\documentclass{article}
\usepackage{amsmath}
\usepackage{amssymb}
\usepackage{cancel}
\usepackage{setspace}
\usepackage{graphicx}
\usepackage{enumitem}
\usepackage[english]{babel}
\usepackage[letterpaper,top=2cm,bottom=2cm,left=1cm,right=2cm,marginparwidth=1 cm]{geometry}
\usepackage{tikz}

\onehalfspacing
\begin{document}
\title{Simgesel Yaklaşımlarla Sayısal İntegral İfadelerinin Türetilmesi}
\date{\today}
\maketitle

\section{Kullanıcıdan alınan n = paramN değerine karşılık ilgili polinomlar oluşturulur.}
$f(x) = paramFx$\\
paramFX$

\section{Koordinat düzlemi üzerinde yer alacak h ve y değerleri belirlenir.}
$h = paramH$\\
$y = paramY$

\section{Polinomlar için x - h dönüşü yapılır.}
paramXToH


\section{Katsayılardan oluşan sembolik matris oluşturulur.}
\begin{center}
$\begin{bmatrix}
prmSymbolicMatrix
\end{bmatrix} $
\end{center}

\section{İndirgenmiş eşelon matrisin oluşturulması için katsayılardan oluşan başlangıç matrisi oluşturulur.}
\begin{center}
$\begin{bmatrix}
prmInitialMatrix
\end{bmatrix} $
\end{center}

\section{Adım adım indirgenmiş matris hesaplanır.}
\begin{center}
prmStepByStep
\end{center}

\section{Kök çözüm değerleri belirlenir.}
paramEquationRootValues

\section{İntegral ifadeside çözüm değerlerin yerine konulması sağlanır. ($c_{i}$ $\leftrightarrow$ $y_{i}$)}
paramResult

\end{document}
