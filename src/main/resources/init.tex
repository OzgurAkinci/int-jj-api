\documentclass{article}
\usepackage{amsmath}
\usepackage{amssymb}
\usepackage{cancel}
\usepackage{setspace}
\usepackage{graphicx}
\usepackage{enumitem}
\usepackage[english]{babel}
\usepackage[letterpaper,top=2cm,bottom=2cm,left=1cm,right=2cm,marginparwidth=1 cm]{geometry}
\onehalfspacing
\begin{document}
\title{Simgesel Yaklaşımlarla Sayısal İntegral İfadelerinin Türetilmesi}
\date{\today}
\maketitle

\section{Kullanıcıdan alınan n = 2 değerine karşılık ilgili polinomlar oluşturulur.}
$f(x) = c_{2}x^2 + c_{1}x + c_{0}$\\
$f(x) = (\frac{c_{2}x^3}{3}) + (\frac{c_{1}x^2}{2}) + ((c_{0})x)$

\section{Koordinat düzlemi üzerinde yer alacak h ve y değerleri belirlenir.}
$h = -h,0,h$\\
$y = y_{-1},y_{0},y_{1}$

\section{x - h dönüşü yapılır.}
$x< - >-h$\\
$c_{2}(-h)^2 + c_{1}(-h) + c_{0}$\\
$x< - >0$\\
$c_{2}(0)^2 + c_{1}(0) + c_{0}$\\
$x< - >h$\\
$c_{2}(h)^2 + c_{1}(h) + c_{0}$


\section{Sembolik matris oluşturulur.}

\begin{center}

$\begin{bmatrix}

h^2&-h&1&y_{-1}\\
0&0&1&y_{0}\\
h^2&h&1&y_{1}\\


\end{bmatrix} $ 

\end{center}


\end{document}